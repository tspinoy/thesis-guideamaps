\chapter{Evaluation}\label{ch:evaluation}

In our tool, usability and user experience are very important. Therefore, we did two user studies. The first was to test the functionality of the GuideaMaps (as map creator and as end-user), while the second tested the use case about Plateforme DD.\\

This chapter is divided in two sections. The first describes the user study of GuideaMaps together with its results. In the second section, the same is done for the user study of Plateforme DD.





\section{GuideaMaps}

\subsection{Tasks \& Questions}

\subsection{Results}





\section{Plateforme DD}
The evaluation of the use case about Plateforme DD is different in comparison to the one about GuideaMaps. A first difference is that our audience is different: for GuideaMaps we chose for participants with a background in Computer Science to take the role as map creator. The participants taking the role of end-user were also more than 20 years old. Now, for Plateforme DD, our audience is between ten and fifteen years old. This is because the content on the website\footnote{\url{https://sciences.brussels/dd/}} was created by children of similar ages \textcolor{red}{(TODO: check this)}. However, the participants of the user study never worked with the website before.

\subsection{Setup}
The children were divided in two groups: one group solved the questionnaire using the website, while the other group solved the same questionnaire using the application. There was one tablet per two children on which the website or the application was running (depending on the group they were in). It was important they did not work with both systems because then there was the possibility a learning effect of having used one system before the other could influence the results. 

\subsection{Tasks \& Questions}
The user study consisted of different tasks and questions the children had to solve. Before they started with the actual tasks and questions, we gave them five to ten minutes to explore the website or the application so that they could get used to it a bit. We mentioned that it was important to understand the structure as much as possible because some questions would follow after this introduction period.\\

After five to ten minutes, we distributed the papers with the tasks and the questions. With other words, all children (of both groups) received part 1 of the questionnaire (see \autoref{appendix:pdd-questionnaire}). \textcolor{red}{TODO: describe questions and tasks}\\



\subsection{Results}



