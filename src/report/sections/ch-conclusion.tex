\chapter{Conclusion \& Future Work}\label{ch:conclusion-future-work}

\section{Conclusion}\label{sec:conclusion}
In this thesis, we examined different visualization techniques in multiple domains. We started with a discussion of existing techniques and tools. It was clear that these were not very suitable for the goals we had in mind. We wanted a tool able to represent linked data and knowledge in a visual manner. Therefore, GuideaMaps was a good starting point. However, GuideaMaps had a number of limitations. First, the tool was created for iPads only. Hence, users were limited to a specific device and operating system. Further, there was no way to change the layout of the nodes and links. This thesis presented a solution to these and other limitations: GuideaMaps 2.0.\\

To achieve the goals (\autoref{sec:research-goals}) and requirements (\autoref{sec:requirements}), some crucial design decisions were made. The goal to have a device- and OS-independent application, we chose for a browser-based solution. In this way, users are not restricted to a particular device or OS anymore. Note that Google Chrome is currently the only browser in which the tool is tested, so other browsers can have a different (unexpected) behaviour.\\

Another difference with the first version of GuideaMaps is that we made it easier to create template maps. In the first version, this had to be done with XML, while we now have a graphical way to do this action. This feature should make it easier for users with less background in Computer Science to create templates.\\

The biggest contribution of this thesis can be found in the genericity of our application. A tool that can be used for much more than only one domain (e.g. requirement elicitation) is a feature most people really like. Therefore, the application is created as a library. Developers can reuse the core of our application and replace the code responsible for the visualization of the nodes and links by his own implementation. His code can simply be plugged in without affecting the general implementation.\\

We strongly believe that this is a feature developers will like and tested the ease to plug in other code by means of a use case, i.e. Plateforme DD. It takes only a minute to plug in a custom implementation and run the library. The use case was also coupled to an evaluation of the visualization tool. Half of the participants worked with a website with a tree structure under the hood, while the other half worked with the same content of the website visualized by our application. The intent of visualizing the content was to make the structure more understandable. The results of the questionnaire confirmed our expectations: searching and finding information is easier when using our application than the website.


\section{Future Work}\label{sec:future-work}

The presented application could be improved in different ways in the future. One of the most important improvements that could be done is supporting a broader range of browsers. Currently, the tool works perfectly in the latest version of Google Chrome. It is good to have a working tool in one of the most widely used browsers, but lots of people use other browsers apart from Google Chrome. We detected some issues in Firefox, where for instance the tool is very slow on a tablet. In the future, we should improve the system such that it works in different browsers and if possible in all commonly used ones.\\

Furthermore, it is essential to be able to store the created templates and maps. Therefore, the system should be further elaborated so that the changes are written to a file that can be loaded the next time the map is opened. Right now, the maps are re-initialized every time the page is refreshed.\\

Next to supporting persistence, it would be useful make the tool collaborative, i.e. allow multiple people to work simultaneously on the same map and see each others (saved) changes in real-time.\\

Another important improvement would be to make it possible to drag and drop the nodes of the visualization to support the similarity and proximity principle of the Gestalt Psychology Theory \citep{koffka2013principles}.\\
