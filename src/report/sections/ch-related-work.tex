\chapter{Related Work}\label{ch:related-work}

There exist lots of ways to visualize data and the relations between data. In this chapter, we discuss some of the possibilities related to what we need for GuideaMaps.

% ---------------------
% ----- MIND MAPS -----
% ---------------------
\section{Mind Maps}
The most well-known technique to visualize related data is to create a mind map (a.k.a. idea map). This technique is mainly used to show the relation between portions of information and for brainstorming purposes. Other applications where this technique is used are note-taking, problem solving, etc. \citep{knowledgemapsbalaid} \\

Mind maps are created by writing the main idea in the middle of the drawing, while all sub-ideas are placed around that center node. Each sub-idea is connected with its parent by means of a line. Hence, this kind of visualization is not difficult to create or understand. Its simplicity is one of the reasons why it is used a lot in practice.\\

Because mind maps is not a new concept, but one that most people already know quite well, we will not discuss it into further detail.





% --------------------------
% ----- KNOWLEDGE MAPS -----
% --------------------------
\section{Knowledge Maps}
According to \cite{knowledgemapsbalaid}, \textit{knowledge maps} is an umbrella term for tools and techniques like mind maps. \cite{knowledgemapsodonnell} defined the concept as follows:

\begin{quote}
Knowledge maps are node-link representations in which ideas are located in nodes and connected to other related ideas through a series of labeled links. \hfill 
\end{quote}

This way of representing information has for example a positive impact on students. The paper of \cite{knowledgemapsodonnell} teaches us that students using knowledge maps are better in remembering the main ideas of the subject in comparison to the ones that study from the text without the visualization. As GuideaMaps's context is more focused on representing large amounts of knowledge in an easy to grasp way, it is less important that the users remember the content of the visualization, but the node-link representation with the main idea in the center also showed to be useful for this purpose, as illustrated by the popularity of mind maps.\\

Next to mind maps, concept maps is a second technique included under the umbrella of knowledge maps. Concept maps are similar to mind maps in some sense but they do have some different characteristics. First, the purpose of a mind map is to associate ideas, topics or things, while concept maps illustrate relations between concepts. Further, the structure of a concept map is mostly hierarchical and visualized like a tree. On the other hand, mind maps sometimes have a radial layout and not hierarchical. \citep{davies} \\

Hence, we can state that GuideaMaps makes use of a knowledge map visualization and more specifically some kind of combination of mind maps and concept maps.