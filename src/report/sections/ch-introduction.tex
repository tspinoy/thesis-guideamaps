\chapter{Introduction}\label{ch:introduction}

\section{Problem statement}\label{sec:problem-statement}
For the first version of GuideaMaps, the main goal was to develop the following:

\begin{quote}
A tool that allows the different people (and with different background) involved in the development of a serious game (e.g., against cyber bullying) to brood over the goals, characteristics and main principles of a new to develop serious game. The tool should be easy to use and usable in meetings. Therefore, we want to explore the characteristics and capabilities of a tablet (i.e. iPad). \hfill \citep{erikjanssens}
\end{quote}

By specifying the goal in this way, end users of the application are restricted to some factors. First, they need an iPad to be able to use the application. It has to be an iPad, and not an other kind of tablet with a different operating system, because GuideaMaps was created and designed for iOS. In some cases this can be a hard restriction, e.g. suppose one of the participants of the meeting does not have an iPad. Then there is no solution for that person to let him use the application. \\

Further, the tool is mainly created for the purpose of serious games. It is not possible to use a slightly different visualization in order to use it for other purposes than a serious game.

\section{Research questions}\label{sec:research-questions}
The problem statement discussed in the previous section indicates that the first version of GuideaMaps comes along with some restrictions. Therefore, the following research questions about an optimized version of the application can be posed:

\begin{description}
	\item[Question 1] \hfill \\
	Can we create a version of GuideaMaps that works on all devices and all different operating systems?
	
	\item[Question 2] \hfill \\
	Can we make that application generic in such a way that it can be used for different purposes?
	If so, is it possible to make it customizable so that the user can define its own layout for the visualization?
\end{description}

In this thesis, we strove to find a solution with which we can tackle the problem and searched for answers on the review questions. This solution is called \textit{GuideaMaps 2.0}.

