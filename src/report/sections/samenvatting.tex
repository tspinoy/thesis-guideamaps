\section*{Samenvatting}\enlargethispage{1.5\baselineskip}
\markboth{Samenvatting}{Samenvatting} % ensure the correct name appears in the header and footer

Vandaag de dag bestaan heel wat tools om verbanden tussen data te visualiseren, e.g. mind map tools. Helaas missen de meeste van deze tools belangrijke functionaliteit, zoals de mogelijkheid om je eigen ``look and feel'' voor je visualisatie te defini\"eren, om templates te maken, of om meer te kunnen dan enkel een mind map maken die je gedachten ordent. In deze thesis wordt GuideaMaps 2.0 voorgesteld, een browser-gebaseerde tool om op een eenvoudige manier kennismappen te maken en die de reeds genoemde functionaliteit niet mist. Met ``eenvoudig'' bedoelen we: niet gelimiteerd zijn tot een bepaald apparaat of platform en over de mogelijkheid beschikken om de tool voor meerdere doeleinden te kunnen gebruiken. Terwijl de eerste versie van GuideaMaps voornamelijk gemaakt werd voor requirement elicitation hebben we het spectrum van situaties waarin de tool gebruikt kan worden verbreed. In een illustrerende use case werd een website met een ingewikkelde boomstructuur gevisualiseerd door onze tool om de structuur duidelijker te maken. In een user study met 52 deelnemers werd het duidelijk dat onze visualisatie eenvoudiger was in gebruik in vergelijking met de website.\\

Een andere belangrijke bijdrage is dat het systeem gemaakt is als een library, i.e. ontwikkelaars kunnen de implementatie van de tool uitbreiden en bewerken als dit nodig is voor het doel dat ze voor ogen hebben. De library laat toe om de visualisatie van de nodes en de links aan te passen zonder de standaard implementatie van de tool te be\"invloeden. Zo een eigen visualisatie kan eenvoudig ingeplugd worden. Verder zijn er twee modi voorzien: (1) map creators kunnen templates maken voor een bepaald doel en (2) eindgebruikers kunnen de templates vullen met de nodige data. We kunnen dus zeggen dat GuideaMaps 2.0 op veel vlakken verschillend is van bestaande tools. Andere systemen zijn vaak bruikbaar voor \'e\'en enkel doel, terwijl onze applicatie probeert om functioneel te zijn in veel verschillende situaties.\\

De standaard GuideaMaps visualisatie werd ge\"evalueerd in een andere user study. De resultaten toonden aan dat zo een tool maken niet triviaal is. Kleine details (e.g. symbolen) kunnen leiden tot frustraties en de tool minder intu\"itief maken.