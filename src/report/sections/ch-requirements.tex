\chapter{Requirements}\label{ch:requirements}

\section{Functional requirements}\label{sec:functional-requirements}
The application should provide a lot of functionalities in order to be useful in many cases. The most important one is that the system should show the current state of the data which is created by the user or loaded via a file. This data should be nicely visualized, easy to interpret and straightforward to edit. A difference with the first version of the GuideaMaps is that the tool should run in the browser. While version 1.0 was only designed for an iPad, a browser based version like GuideaMaps 2.0 immediately is a tool that can be used on multiple devices (e.g. tablets, laptops and desktops) and different operating systems. Hence, the user is now not limited to a particular device anymore, the tool can be used on any device with an internet connection.

\section{Usability requirements}\label{sec:usability-requirements}
In order to let a user create, explore and edit the data, the usability of the application should be as high as possible. This is an important requirement because the tool will not only be used by people with experience in Computer Science. It doesn't matter whether or not the user has a background in Computer Science, he should be able to easily learn the system in short time. Therefore, it is important to choose for clear, not misunderstandable icons on buttons where a click on this button invokes a certain action. Some examples of these actions are (1) adding a child node, (2) edit a node and (3) expand/collapse a node. Good icons for each of the mentioned actions are shown in figure \ref{fig:icons}.

\begin{figure}[H]
	\centering
	\begin{subfigure}{.2\textwidth}
  		\centering
  		\includegraphics[width=.25\linewidth]{plusicon}
  		\caption{Plus}
  		\label{fig:plusicon}
	\end{subfigure}%
	\begin{subfigure}{.2\textwidth}
  		\centering
  		\includegraphics[width=.25\linewidth]{editicon}
  		\caption{Edit}
  		\label{fig:editicon}
	\end{subfigure}
	\begin{subfigure}{.2\textwidth}
  		\centering
  		\includegraphics[width=.25\linewidth]{expandicon}
  		\caption{Expand}
  		\label{fig:expandicon}
	\end{subfigure}
	\begin{subfigure}{.2\textwidth}
  		\centering
  		\includegraphics[width=.25\linewidth]{collapseicon}
  		\caption{Collapse}
  		\label{fig:collapseicon}
	\end{subfigure}
	\caption{Good icons for the following actions: (a) add child node, (b) edit node, (c) expand node and (d) collapse node.}
	\label{fig:icons}
\end{figure}